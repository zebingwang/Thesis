%---------------------------------------------------------------------------%
%->> Backmatter
%---------------------------------------------------------------------------%
\chapter[致谢]{致\quad 谢}\chaptermark{致\quad 谢}% syntax: \chapter[目录]{标题}\chaptermark{页眉}
%\thispagestyle{noheaderstyle}% 如果需要移除当前页的页眉
%\pagestyle{noheaderstyle}% 如果需要移除整章的页眉

时光飞逝,转眼间就到了博士毕业的时候了,回想五年前刚刚步入中国科学院高能物理研究所读研,一切仿佛都发生在昨天。在这期间,我体会到了除了家人以外带给我的温暖,感谢老师、家人和朋友们在这五年期间对我的帮助和支持。

首先我要感谢我的博士生导师陈明水研究员,感谢陈老师在这五年内对我的悉心指导。在生活方面,陈老师待人温和,对我们学生的生活关爱有加,也为我们提供了很多有用的建议。在科研方面,陈老师要求严格,亲身加入到每个学生的日常工作中去,深入了解每个学生的科研进展,并为我们研究中遇到的困难出谋划策,帮助我在这五年时间里能够顺利完成我的博士论文。令我印象最为深刻的是陈老师几乎每天都会组织学生参加线上讨论,期间听取各个学生的科研进展,并对遇到的问题进行讨论。这让我深深明白了交流的重要性,特别是在这种大型国际合作的项目中,充分的交流可以提高工作效率,同时它也让我能够对其他方向的科研进展有所了解,对我今后的科研生活产生了非常重要的影响。除此之外,在我博士期间我也遇到了非常多的困难和挫折,非常感谢陈老师在这期间对我的鼓励和指点,让我能够克服这些困难。另外,也感谢陈老师推荐我申请国家公派留学到CERN进行学习工作,在此期间,我学习到了非常多的知识,并且培养了我的交流能力,让我受益匪浅。

感谢高能所CMS组的老师陈国明研究员、张华桥研究员、陶军全副研究员、廖红波副研究员、王锦副研究员和陈晔副研究员、王峰助理研究员,虽然没有直接指导我的科研工作,但是当我寻求帮助时他们总是耐心作答,给了我非常多的建议,我对此非常感谢。

感谢高能所HGCal实验室,让我能够在博士学习期间获得有关实验方面的经验。尤其要感谢王峰助理研究员,在我的实验过程中对我所遇到的困难提供了非常大的指导与帮助。

感谢CERN的CSC课题组,让我在CERN的交换期间可以接触到有关探测器方面的工作,极大的拓宽了我的研究视野。尤其要感谢Katerina Kuznetsova和王建老师,感谢他们在我参与CSC项目时对我提供的指导与帮助。

感谢我的师兄师姐郭倩颖、张辰光、余涛哲、彭娜,博后Ram Krishna Sharma,同学王储、苑超辰、华慧玲,大家不仅在生活中互相帮助,更会在科研中互相讨论,给了我非常多的建议和帮助。

感谢国科大和中国留学基金委对我出国联合培养的资助,在CERN交流学习的这两年,让我接触到了最前沿的学术知识,锻炼了我的了流能力,令我受益匪浅。

感谢我的女朋友马少杰,我们相识于日内瓦,在这异国他乡,你成了我最大的港湾。转眼间我们已经恋爱一年了,感谢你在这期间对我的包容、理解和支持,感谢你在我低谷时期带给我的开心与快乐,让我能够顺利完成论文。

最后我要感谢我的父母,感谢你们给了我足够的支持和理解,让我能够拥有一个宽松自由的环境完成我的学业,感谢你们从小对我的养育和培养,让我能够最后走上物理这条道路。


\rightline{2023年8月}
\chapter{作者简历及攻读学位期间发表的学术论文与其他相关学术成果}

\section*{作者简历:}
2014年9月——2018年6月,在湘潭大学物理与光电工程学院获得学士学位。

2018年9月——2023年8月,在中国科学院高能物理研究所攻读博士学位。


\section*{已发表(或正式接受)的学术论文:}

{
\setlist[enumerate]{}% restore default behavior
\begin{enumerate}[nosep]
    \item 主要作者:Search for the exotic decay of the Higgs boson into a Z boson and a light pseudoscalar decaying into two photons in pp collisions at $\sqrt{S} = 13~\si{TeV}$. CMS-PAS-HIG-22-003, 2023, URL: https://cds.cern.ch/record/2853524.
    \item 参与者:Response of a CMS HGCal silicon-pad electromagnetic calorimeter prototype to 20–300 $\si{GeV}$ positrons [J/OL]. Journal of Instrumentation, 2022, 17(05): P05022. DOI: 10.1088/1748-0221/17/05/P05022
\end{enumerate}
}

\section*{国际会议海报与报告:}

海报:“Search $\HZa$ at LHC” at “2023 Large Hadron Collider Physics Conference (LHCP), Serbia, Belgrade, 22–26 May, 2023”

海报:“Longevity Studies and Eco-friendly Gas Searches for CMS Cathode Strip Chambers” at “2023 CMS Upgrade Days, CERN, Geneva, 6–8 Feb, 2023”

报告:“CMS-CSC Longevity and Ecogas Search” at “2022 Summer Muon Week, CERN, Geneva, 23 Jun 2022”


\cleardoublepage[plain]% 让文档总是结束于偶数页,可根据需要设定页眉页脚样式,如 [noheaderstyle]
%---------------------------------------------------------------------------%
