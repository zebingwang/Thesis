%%% Local Variables: 
%%% mode: latex
%%% TeX-master: t
%%% End: 

\chapter{总结与展望}

本论文主要论述了利用希格斯粒子的奇异衰变过程来对类轴子进行寻找,分析中使用了位于大型强子对撞机上的CMS探测器于2016、2017和2018年间收集到的数据,对应的质心能量为13~\si{\TeV},总计积分亮度为138~\si{fb^{-1}}。在本分析中,希格斯玻色子会奇异衰变到一个Z玻色子和一个类轴子,其中Z玻色子衰变到两个轻子,类轴子衰变到两个光子。考虑到对类轴子衰变产生的两个光子重建的可行性,本论文主要对质量位于1--30$\GeV$内的类轴子进行了寻找。针对低质量类轴子的情况,本论文研究了CMS实验官方的光子辨别算法,发现算法中光子的$\sigma_{i\eta i\eta}$和$I_{\gamma}$会对信号样本的选择效率产生非常大的影响。因此,本论文设计了新的光子辨别算法,去除了$\sigma_{i\eta i\eta}$和$I_{\gamma}$这两个变量的影响,最终使得低质量类轴子区间内的信号选择效率恢复到了同高质量类轴子区间相同的水平。由新设计的光子辨别算法所带来的数据和蒙卡样本之间的差异可以使用由tag-and-probe方法计算得到的比例因子进行修正。为了能够完整重建出希格斯玻色子,本论文要求事例中至少存在两个带电相反味道相同的轻子和两个光子。经过筛选后的事例中的主要本底过程来自于Drell-Yan过程所产生的一个Z玻色子和众多的喷注,其中Z玻色子衰变到一对轻子,喷注被误判为光子。为了提高信号的显著度,本论文采取了多变量分析的方法,利用了光子、类轴子和希格斯粒子之间的动力学信息作为输入,对信号和本底进行了区分。除此之外,为了使得训练出来的分类器能够在整个类轴子区间内都保持比较高的分类性能,本论文采取了参数化的决策树林法来对分类器进行训练,将不同质量点的信号样本作为参数输入到训练中,进而可以得到能够应用于整个类轴子质量范围的分类器模型。通过使用训练出来的分类器,可以在保持信号效率达到80\%的条件下拒绝掉99.9\%的本底事例。而后,通过分类器筛选的事例被用于最终的信号提取,提取过程采取了轮廓似然比的方法,通过使用参数化的信号加本底模型直接对数据中的$\mllgg$分布进行拟合得到对应的信号强度上限。其中,所使用的信号模型通过拟合信号蒙卡样本得到,拟合模型采用了n个高斯函数之和;本底函数模型通过使用包络线的方法直接从数据中拟合得到。由于本论文中要求事例中包含一个在壳的Z玻色子和两个横动量大于10$\GeV$的光子,使得$\mllgg$的不变质量谱会在110$\GeV$附近出现一个turn-on。为了描述这一部分本底形状,最终的本底拟合范围选择为95--180$\GeV$,并且使用了一个高斯函数卷积上一个下降谱和阶梯函数的乘积进行拟合,高斯部分用来描述本底中的turn-on成分。由于我们对数据中的本底分布形状没有任何的先验信息,本底拟合选择了多种拟合函数形式。其中,由本底函数的选择所造成的偏差可以使用偏差评估进行估计,结果表明所有质量点的偏差都小于14\%,对应的对最终上限的计算影响小于1\%。因此,由不同本底函数的选择所造成的偏差可以忽略不计。最终,本论文给出了置信度为95\%的$\HZa$截面测量上限,这是对希格斯粒子衰变到双光子和双轻子末态的首次测量,1$\GeV$和30$\GeV$对应的观测(预计)到的反应截面上限为18.9(19.3)\si{fb}和4.8(6.9)\si{fb}。没有发现明显超出预期的观测结果,仅在3$\GeV$质量点处观测到了2.6倍$\sigma$的局域显著度,但考虑到Look-Elsewhere-Effect的全局显著度仅为1.3倍$\sigma$。除此之外,本论文还将结果应用到了对类轴子模型的解释,计算过程中假设类轴子衰变到双光子的分支比为100\%,最终在95\%的置信度下对希格斯粒子和类轴子之间的耦合参数$\cZh$的上限进行了测量。测量结果排除了耦合强度大于0.1~\si{\TeV^{-1}}的范围,这是利用希格斯粒子的双轻子加双光子末态对类轴子的首次测量。

除了上述分析,本论文还讲述了作者攻读博士期间参与的硬件工作项目,主要包括HGCal模块的组装生产和CMS实验中CSC探测器寿命的研究。对于HGCal项目,本论文展示了作者参与并完成了第一块8英寸HGCal模块的组装生产工作;对于CSC探测器项目,本论文展示了对CSC探测器中不同$\CF$含量的研究情况,并得出结论认为2\%的$\CF$含量可能存在一定风险,目前正在对$\CF$的含量为5\%的情况进行研究。

在对本论文所论述的分析工作进行总结之后,可以展望一下未来能够将分析进一步优化的可能性。由于本分析是对希格斯粒子的双轻子加双光子末态的首次测量,因此并没有考虑末态光子中存在一个低动量光子的情况。根据研究发现,很大一部分信号事例的末态光子中存在一个横动量小于10$\GeV$的光子,此时这个光子的横动量小于CMS探测器中对光子重建所需的最低横动量要求,因此没有办法完全重建出两个末态光子,使得信号选择效率普遍偏低;此外,对于低质量的类轴子,末态衰变产生的两个光子没有办法在探测器中分辨出来,从而也不能够完全重建出两个光子。针对上述情况,目前CMS实验研发了一种基于端到端的深度学习技术~\cite{cms2022search},可以在探测器中利用光子在量能器中的能量沉积图像直接通过图像识别辨别出两个融合的光子。这项技术同时对其中一个光子横动量小于10$\GeV$的情况也具有非常好的辨别能力。通过使用这项技术,我们可以在未来进一步对希格斯粒子衰变到双光子加双轻子的末态进行进一步的限制。同时,我们也可以利用这项技术寻找更低质量的双光子共振态,探寻更大范围的类轴子。

除此之外,在未来HL-LHC运行阶段可以产生的瞬时亮度是目前的三倍,对应的可收集到的数据统计量是目前的10倍,可以大大提升对类轴子的探测灵敏度。而在经过第二次长期停机升级之后,CMS探测器对各种粒子的鉴别能力都会有所提升。对缪子而言,升级之后的探测器会在前向区域($1.6 < |\eta| < 2.4$)添加额外的缪子探测器以及将探测缪子的覆盖范围扩展到$|\eta| = 3$的位置,这样可以使得CMS探测器在本底比较复杂的前向区域内获得更高的缪子重建能力;而对于电子和光子,升级后的HGCal探测器会为端盖处的电子和光子的辨别额外提供精细的纵向簇射形状信息,提高电子和光子的辨别效率。所有这些升级都会使得我们在未来对类轴子的探测灵敏度获得提升。